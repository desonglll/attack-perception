\section{拟采取的研究方案}

\subsection{基于机器学习的工业互联网络智能攻击检测算法}

\subsubsection{工业互联网的攻击行为及攻击原理相关的方法}

工业互联网的攻击行为因其复杂性和目标的多样性,通常需要结合多个维度进行研究。文献中常见的方法包括基于威胁建模和行为分析的手段。威胁建模方法如STRIDE模型,通过分类分析攻击目标的潜在威胁,包括欺诈、伪造、信息泄露和服务拒绝等;行为分析方法则通过收集网络流量和设备日志数据,提取攻击特征用于深入解析攻击链。近年来,图结构分析(如攻击图和依赖图)被用来研究攻击路径及其传播特性,有助于揭示多阶段攻击的演化过程。此外,动态模拟方法(如蜜罐系统)常用于诱捕攻击者并记录其行为模式,为构建更全面的防御机制提供支持。文献还指出,攻击行为的多样化和隐匿性使得对异常行为的深度学习建模成为一种关键技术趋势。

\subsubsection{工业互联网攻击行为检测的方法}

工业互联网攻击行为的检测方法分为基于签名和基于异常的检测。签名检测依赖于已知攻击特征的规则库,通过匹配流量特征来识别攻击行为,常用于入侵检测系统(IDS)。异常检测则利用机器学习技术,分析正常网络行为的模式并识别偏离的异常行为。例如,支持向量机(SVM)、随机森林(RF)等传统机器学习算法在检测工业控制系统中的异常流量方面表现出色。近年来,深度学习模型(如卷积神经网络CNN、长短时记忆网络LSTM)被广泛应用于网络流量分类和时间序列预测,显著提高了检测精度。此外,联合多任务学习与特征选择的混合模型正在兴起,可在降低误报率的同时提升对未知攻击的检测能力。一些研究还提出了基于边缘计算的分布式检测架构,提高了系统对实时攻击的响应能力。

\subsection{复杂网络攻击下的工业互联网络动态风险评估方法}

\subsubsection{攻击行为对工业互联网影响的描述方法}

在工业互联网领域,对攻击行为影响的描述方法研究具有重要意义,首先在网络层攻击溯源技术层面,为解决溯源难题而提出的分层式无状态单分组 IP 溯源技术是关键。该技术从域间和域内两个粒度着手,能够精准重构攻击路径,进而精确确定攻击源。尤其在面对节点摘要信息篡改和伪造情况时,展现出卓越的健壮性,为网络层攻击溯源提供了可靠的解决方案。其次,在应用层洪泛攻击场景中,依据大偏差理论提出的用户访问行为相似性度量方法发挥了重要作用。它不仅可以准确描述用户访问行为的相似性,而且在应用层洪泛攻击检测实践中得到了有效验证,为攻击检测提供了新的思路和方法。最后,对于面向应用层分布式拒绝服务攻击检测模型,以访问会话长度、访问对象热度等为特征,并运用聚类算法进行检测。这种方式能够达到 90\% 的检测率,为应对此类攻击提供了有力的技术支持。


\subsubsection{工业互联网的风险评估内容评价方法}

工业互联网风险评估是一个复杂而多维的过程,涉及对潜在威胁、资产价值以及脆弱性的综合考量。首先,资产识别与评估是风险评估的基石。这一过程涵盖了硬件(如服务器、网络和工业控制设备)、软件(包括操作系统、工业控制和应用程序)以及数据(生产数据、工艺参数、用户信息等)的分类与价值评估。其次,威胁识别与分析是评估过程中的关键环节。威胁源可分为外部(如黑客、网络犯罪组织、竞争对手)和内部(如员工误操作、恶意内部人员)两大类。威胁可能性的评估基于历史数据、行业趋势和当前安全防护状态,用概率值或等级(高、中、低)表示。脆弱性评估则包括漏洞扫描和配置审查。漏洞扫描旨在全面扫描系统和设备的网络、软件、工业协议漏洞,使用专业工具发现薄弱点。最后,评价方法的动态性和持续性是确保风险评估时效性和有效性的关键。这包括定期评估,根据环境变化调整评估周期,以及事件驱动评估,即在重大安全事件或系统、外部环境变化时及时启动评估。


\subsection{攻击行为驱动的工业互联网络协同安全防护策略}

\subsubsection{工业互联网的安全防御策略方法}

攻击行为驱动的工业互联网络协同安全防护策略,旨在构建一个以攻击行为为导向、多层级协同的安全防护体系。该策略的核心安全防御方法包括:首先,强化数据加密与隐私保护,采用先进的加密技术确保敏感数据在传输和存储过程中的安全,同时建立健全的隐私保护机制,防止数据泄露。其次,构建多层次的网络安全防线,通过使用防火墙、入侵检测系统等安全工具,结合实时安全监测和预警系统,提高对网络攻击的防御和响应能力。最后,加强物联网设备的安全管理,包括全面的安全评估、漏洞修补、权限管理和固件更新等,确保物联网设备的稳定与安全运行。这些策略的实施需要企业、科研机构等多方协同合作,共同提升工业互联网的安全防护水平。


\subsubsection{工业互联网安全防御策略方法之间的协同方法}

攻击行为驱动的工业互联网络协同安全防护策略,其核心在于多种安全防御策略方法之间的协同工作。以下是对协同方法的重点描述:首先,不同安全防御策略方法之间要实现信息共享与交互。例如,防火墙、入侵检测系统(IDS/IPS)等安全设备可以实时共享网络流量、异常行为等安全信息,以便协同响应攻击行为。其次,通过构建协同防御框架,实现客户机、网络监控平台与安全防御系统之间的协同工作。这种协同不仅限于单一层面的防御,而是要在网络、应用、数据等多个层面实现全面协同。最后,建立统一的安全管理平台,对多种安全防御策略方法进行集中管理和调度。该平台可以根据攻击行为的特征和趋势,动态调整安全策略,实现协同防御的最优化。


\subsection{面向车辆网(农机互联网)的动态安全防护应用验证}

(1)车辆网仿真平台搭建思路

面向车辆网(农机互联网)的动态安全防护应用验证的车辆网仿真平台搭建思路,旨在模拟真实的车辆网环境,以验证安全防护策略的有效性和可靠性。以下是具体的搭建思路:

首先,明确仿真平台的目标和需求。这包括确定要模拟的车辆网规模、节点类型、通信协议以及安全防护策略的具体要求。基于这些目标和需求,进行平台架构设计,选择合适的硬件和软件组件,以确保仿真平台的稳定性和可扩展性。
其次,在平台中构建车辆网模型。这包括模拟车辆节点、传感器节点、通信链路等,并配置相应的网络拓扑和通信协议。同时,还需模拟攻击行为,以验证安全防护策略在应对各种攻击时的表现。

接着,集成安全防护策略。将待验证的安全防护策略集成到仿真平台中,包括数据加密、身份认证、访问控制等。这些策略应根据车辆网的特点和需求进行定制和优化。

最后,进行仿真测试和验证。通过模拟真实的车辆网运行场景和攻击行为,对仿真平台进行测试和验证。收集和分析测试结果,评估安全防护策略的有效性和可靠性,并根据需要进行调整和优化。

综上所述,搭建面向车辆网的仿真平台需要综合考虑目标需求、平台架构、模型构建、安全防护策略集成以及仿真测试和验证等多个方面。

(2)车辆网实物平台验证思路

针对攻击行为驱动的工业互联网络协同安全防护策略,在面向车辆网(特别是农机互联网)的动态安全防护应用验证中,车辆网实物平台验证思路可以概括为以下几点:

首先,需要构建一个高度仿真的车辆网实物平台,该平台应能够模拟真实农机互联网环境中的网络通信、数据传输以及设备交互等场景。这包括模拟农机的各类传感器、控制器和执行器等设备,以及它们与云端服务器的通信过程。

其次,在平台上部署协同安全防护策略,包括数据加密、身份认证、访问控制等安全措施,以及针对特定攻击行为的防御机制。这些策略应能够实时响应并防御来自网络的各种攻击,如数据篡改、重放攻击、拒绝服务等。

接着,通过模拟不同的攻击场景,对平台的动态安全防护能力进行验证。这包括测试平台在遭受攻击时的响应速度、防御效果以及恢复能力等方面。通过不断调整和优化安全防护策略,提升平台对攻击的防御能力。

最后,收集并分析验证过程中的数据,包括攻击行为特征、防御策略效果等,以便对协同安全防护策略进行持续改进和优化。同时,这些数据还可以为未来的安全防护研究和应用提供有价值的参考。
