\section{拟解决的关键科学问题}

\subsection{脆弱网络环境下的工业互联网络智能攻击感知}

智能攻击感知技术是基于机器学习和数据分析的方法,用于实时监测和识别针对信息物理系统(特别是基于CAN总线的工业互联网络)的智能攻击。这种技术通过分析网络流量、设备状态和系统日志等数据,能够自动学习和识别攻击模式,及时发出警报并采取相应的防御措施。其核心在于其能够自动学习和适应不断变化的攻击手段。它利用机器学习算法对大量历史数据进行分析,提取出与攻击相关的特征,并构建出攻击模式库。在实际运行过程中,该技术会实时监测网络流量和设备状态,与攻击模式库进行比对,从而识别出潜在的智能攻击。

智能攻击感知技术框架是一个多层次、模块化的系统,包括数据采集与处理、特征提取与选择、模型训练与检测等多个关键组件。这些组件共同协作,实现了对网络环境中智能攻击的实时监测和识别。以下是对智能攻击感知技术框架的各个组件的详细介绍:

数据采集是智能攻击感知技术的第一步。为了全面获取网络状态和设备信息,需要采集的数据包括网络流量数据、设备状态数据、系统日志等。这些数据通过传感器、网络监控设备等手段进行收集,并经过预处理以提高数据质量。预处理过程包括数据清洗、去噪、归一化等步骤,以确保数据的准确性和一致性。

特征提取是智能攻击感知技术的关键步骤之一。它旨在从预处理后的数据中提取出与智能攻击相关的特征。这些特征可能包括通信模式的变化、数据完整性的破坏、异常的网络流量等。为了筛选出最具贡献的特征,可以采用特征选择方法,如主成分分析(PCA)、互信息(MI)等。

模型训练是智能攻击感知技术的核心环节。它利用机器学习算法对提取出的特征进行训练,构建出攻击检测模型。常用的机器学习算法包括支持向量机(SVM)、神经网络(NN)、决策树等。在训练过程中,需要不断调整模型参数以提高检测性能。训练完成后,模型会部署到实际系统中进行实时监测。在实际监测过程中,智能攻击感知技术会不断接收新的数据,并将其输入到训练好的模型中。模型会根据数据的特征进行实时判断,并输出检测结果。如果检测到潜在的智能攻击,系统会立即发出警报,并采取相应的防御措施,如隔离受感染设备、重启系统等。

尽管智能攻击感知技术在提高系统安全性方面具有显著优势,但也面临一些技术挑战。例如,随着攻击手段的不断演变和复杂化,如何保持模型的更新和适应性成为了一个重要问题。为了解决这一问题,可以采用在线学习技术,使模型能够在实际运行过程中不断学习和更新。此外,还可以引入多模型融合的方法,结合不同模型的优点来提高检测性能。另外,数据隐私和安全性也是智能攻击感知技术需要关注的问题。在数据采集和处理过程中,需要采取适当的加密和脱敏措施来保护用户隐私和数据安全。同时,在模型训练和检测过程中也需要加强安全防护,防止恶意攻击者利用漏洞进行攻击。


\subsection{复杂网络攻击下的工业互联网络动态风险评估}

在复杂网络攻击环境下,工业互联网络动态风险评估问题的核心在于如何准确识别和量化网络攻击对工业控制系统(ICS)的影响,并根据攻击的变化动态调整风险评估模型。随着信息和通信技术的采用,ICS变得非常容易受到网络攻击,动态网络安全风险评估(CSRA)在ICS的安全保护中发挥着至关重要的作用。动态风险评估需要考虑网络攻击的多变性、攻击路径的复杂性以及攻击对ICS具体影响的不确定性。

在动态风险评估中,一个关键的挑战是如何降低建模过程的复杂性。研究者提出了基于关联分析的CSRA方法,设计了一个三层关联网络(AN)来推断安全事件的概率,通过挖掘历史攻击记录的数据得出AN的参数,并通过对系统过程数据的距离相关性分析,量化系统的网络安全风险。此外,由于缺乏足够的历史数据,建立ICS的风险传播模型变得困难,因此提出了一种用于动态风险评估的模糊概率贝叶斯网络(FPBN)方法。这种方法通过分析和预测网络安全风险的传播,并开发了一种近似动态推理算法,用于动态评估ICS网络安全风险,并嵌入了噪声证据滤波器,以减少系统故障引起的噪声证据的影响。

工业控制系统功能安全和信息安全一体化风险评估方法也是一个重要的研究方向。该方法提出了一个集成风险评估算法,包括安全和安全集成风险数据收集、风险分析和风险评估三个步骤。该算法从风险数据的源头出发,同时收集功能安全和信息安全风险数据,生成可以分析网络物理协调攻击路径的扩展攻击树模型,并在计算事件风险时考虑由安全事件和安全事件引起的功能安全损失和信息安全损失,实现功能安全和信息安全的综合风险评估。


\subsection{攻击行为驱动的工业互联网络协同安全防护策略设计}

在设计攻击行为驱动的工业互联网协同安全防护策略时,需综合考量资产识别与评估、威胁识别与分析、脆弱性评估等多个维度,并结合动态性和持续性的评价方法。该策略的核心在于全面识别和评估工业互联网中的硬件、软件和数据资产,以及识别外部和内部威胁源。通过漏洞扫描和配置审查工具,可以发现系统和设备的薄弱点,进而采取相应的安全措施。策略强调定期评估和事件驱动评估的重要性,以确保风险评估的时效性和有效性。在此基础上,构建集成多种应用的工业网络防护平台,利用威胁情报模拟攻击,测试和评估防护系统的检测和防御效果,并以AI技术赋能工业网络安全态势感知。此外,还需结合技术手段和管理体系,建立工业互联网安全管理体系,并与外部实体建立可信的通信通道,实现信息共享与协作。通过这些措施,可以构建一个覆盖安全业务全生命周期的动态防御体系,有效预警、检测和响应安全事件。
